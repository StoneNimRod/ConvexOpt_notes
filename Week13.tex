\documentclass[answers]{exam}

% ------------------------------------------------------------------------------ %
% -----------------------      Base for every .tex file   ---------------------- %
% ------------------------------------------------------------------------------ %

\usepackage[dvipsnames]{xcolor}
\usepackage{mathtools}
\usepackage{amssymb}
\usepackage{amsthm}
\usepackage{amsmath}
\usepackage{framed}
\usepackage{wasysym}
\usepackage{geometry}
\usepackage{cancel}
\usepackage{blindtext}
\usepackage{pgfplots}
\usepackage{graphicx}
\usepackage{lastpage}
\usepackage[most]{tcolorbox} 
\usepackage{multicol}
\usepackage{soul}
\usepackage{listings}
\usepackage{algorithm}
\usepackage{algorithmic}
\usepackage{booktabs}
\usepackage{tikz}
\usepackage{pifont}

% Libraries
\usetikzlibrary{shapes,shapes.geometric, positioning, arrows}

\geometry{%
	left=15mm,
	right=15mm,
	top=25mm,
	bottom=25mm,
	bindingoffset=0mm,
	headheight=30pt,% output from geometry tells you what this needs to be set to as a minimum
}

% Header and Footer
\pagestyle{headandfoot}
\firstpageheadrule
\runningheadrule
\firstpageheader{Convex Optimization}{\today}{Jonathan Schnell}
\runningheader{Convex Optimization}{}{Jonathan Schnell}
\firstpagefooter{}{Page \thepage\ of \numpages}{}
\runningfooter{}{Page \thepage\ of \numpages}{}

% Commands
\newcommand{\imp}[1]{\ul{\textbf{#1}}}
\newcommand{\dproduct}[1]{\left\langle #1 \right\rangle}
\newcommand{\norm}[1]{\left\lVert #1 \right\rVert}
\renewcommand{\vector}[1]{\begin{pmatrix} #1 \end{pmatrix}}
\newcommand{\abs}[1]{\left| #1 \right|}
\newcommand{\floor}[1]{\lfloor #1 \rfloor}
\newcommand{\ceil}[1]{\lceil #1 \rceil}
\newcommand{\fracpart}[2]{\frac{\partial #1}{\partial #2}}
\newcommand{\set}[2]{\left\{#1 \ \middle|\ #2\right\}}
\renewcommand{\hat}[1]{\widehat{#1}}

\newcommand{\Ker}{\operatorname{Ker}}
\renewcommand{\Im}{\operatorname{Im}}
\renewcommand{\Re}{\operatorname{Re}}
\renewcommand{\dim}{\operatorname{dim}}
\renewcommand{\div}{\operatorname{div}}
\newcommand{\rot}{\operatorname{rot}}
\newcommand{\grad}{\operatorname{grad}}
\newcommand{\vol}{\operatorname{vol}}
\newcommand{\supp}{\operatorname{supp}}
\renewcommand{\div}{\operatorname{div}}
\newcommand*{\vertbar}{\rule[-1ex]{0.5pt}{2.5ex}}
\newcommand*{\horzbar}{\rule[.5ex]{2.5ex}{0.5pt}}

\theoremstyle{definition}
\newtheorem*{definition}{Definition}
\newtheorem*{beispiel}{Beispiel}
\newtheorem*{remark}{Remark}

\theoremstyle{plain}
\newtheorem*{proposition}{Proposition}
\newtheorem*{satz}{Satz}
\newtheorem*{korollar}{Korollar}
\newtheorem*{lemma}{Lemma}
\newtheorem*{theorem}{Theorem}


% Quote
\newtcolorbox{zitat}[1]{%
	colback=lightGray,
	grow to right by=-10mm,
	grow to left by=-10mm, 
	boxrule=0pt,
	boxsep=0pt,
	breakable,
	enhanced jigsaw,
	borderline west={4pt}{0pt}{gray},
	#1
}

% Use colors in equations
\newcommand{\highlight}[2]{\colorbox{#1}{$#2$}}%
\definecolor{lightGray}{gray}{0.9} 

% To add shortcut of script Letters in Equations
\newcommand{\s}[1]{\mathcal{#1}}
\newcommand*\circled[1]{\tikz[baseline=(char.base)]{
            \node[shape=circle,draw,inner sep=2pt] (char) {#1};}}
\newcommand{\cmark}{\ding{51}}
\newcommand{\xmark}{\ding{55}}


\newenvironment{claim}[1]{
		\par\noindent
		\textbf{Claim.} #1
		\begin{tcolorbox}[blanker, top=3mm, bottom=3mm, left=3mm, borderline west={1pt}{0mm}{black}]
		\noindent\textit{Proof of Claim.} 
}{
	\hfill$\blacksquare$	
	\end{tcolorbox}\noindent
}

% To add shortcut of number's set Z
\newcommand*{\Z}{\mathbb{Z}}
\newcommand*{\N}{\mathbb{N}}
\newcommand*{\R}{\mathbb{R}}
\newcommand*{\Q}{\mathbb{Q}}
\newcommand*{\C}{\mathbb{C}}
\newcommand*{\F}{\mathbb{F}}
\newcommand*{\K}{\mathbb{K}}

% To add shortcut of empty set
\renewcommand*{\o}{\varnothing}
\pgfplotsset{compat=1.9}

\everymath{\displaystyle}

% Line-Height
\linespread{1.15}

\graphicspath{{Files/}}

% ------------------------------

\begin{document}

	$ $
	\begin{center}
		\huge \textbf{Exercise session notes - Week 13}  \\ \vspace*{3mm}
        \Large{Log-Log convexity + Geometric Programs}
	\end{center}
	$ $\\

    First we reviewed the theory behind log- and log-log-convexity. 
    \begin{definition}
        A function $f:X\to \R$ is called \imp{log-convex} if $f(x) > 0$ and $\log f$ is convex. \\ 
        A function $f:X\to \R$ is called \imp{log-concave} if $1/f$ is log-convex. 
    \end{definition}

    \begin{proposition}
        A function $f$ is log-convex iff $f > 0$ and 
        $$ f\left(tx + (1-t)y\right) \leq f(x)^t \cdot f(y)^{1-t}\qquad \forall x,y,t $$
    \end{proposition}
    \begin{remark} $\ $
        \begin{itemize}
            \item If $f$ is log-convex then $f$ is convex 
            \item Sum $f_1+f_2$, product $\alpha f$, product $f_1\cdot f_2$ of log-convex functions are log-convex 
            \item Affine $a^\top x + b$ is log-concave 
            \item Powers $x^a$ is log-convex for $a\leq 0$ and log-concave for $a \geq 0$
            \item Exponentials $e^{ax}$ is log-affine 
            \item Determinant $\det(X)$ is log-concave
        \end{itemize}
    \end{remark}

    \begin{definition}
        A function $f:X\to \R$ is \imp{log-log-convex} if $f > 0$ and $\log f(e^{x_1}, \ldots, e^{x_n})$ is convex.
    \end{definition}
    \begin{proposition}
        A function $f$ is log-log-convex iff $f> 0$ and 
        $$ f\left(x^t\circ y^{1-t}\right) \leq f(x)^t \cdot f(y)^{1-t} $$
        where the product and the powers on the left are pointwise, i.e.
        \begin{align*}
            x^t\circ y^{1-t} = \begin{pmatrix}
                x_1^t \cdot y_1^{1-t} \\ 
                \vdots \\ 
                x_n^t \cdot y_n^{1-t}
            \end{pmatrix}
        \end{align*}
    \end{proposition}
    \begin{proposition}
        A function $f$ is log-log-convex iff the log-log epigraph 
        $$ \log \operatorname*{epi} f = \set{(\log x,\log t)}{f(x) \leq t} $$
        is a convex set.
    \end{proposition}

    \begin{remark} $\ $
        \begin{itemize}
            \item Posynomials are log-log-convex 
            \item Maximum $\max\{x_i\}$ is log-log-convex 
            \item $L_p$-Norms are log-log-convex
        \end{itemize}
    \end{remark}

    Now we are able to define a Geometric Program. A \imp{monomial} is a function defined as  
    $$ \textbf{x} \mapsto c\cdot x_1^{\alpha_1}\cdots x_n^{\alpha_n} $$
    with $c > 0$ and $\alpha_i \in \R$. A \imp{posynomial} is a function of the form 
    $$ \textbf{x} \mapsto \sum_j c_j\cdot x_1^{\alpha_1^j}\cdots x_n^{\alpha_n^j} $$
    with $c_j > 0$ and $\alpha_i^j \in \R$, so a posynomial is just any sum of monomials. Finally a Geometric program has the form
    \begin{align*}
        \min \qquad f(x)& \\ 
        \text{s.t.} \qquad g_i(x)& \leq 1\qquad \forall i \\ 
        h_j(x) &= 1\qquad \forall j 
    \end{align*}
    where $f, g_i$ are posynomials and $h_j$ are monomials. An important property of GPs is the following 
    \begin{proposition}
        Any Geometric Program can be casted into an equivalent convex program by a change of variables.
    \end{proposition}

    \subsubsection*{Exercise 11.2 (Model Wheel Chair Ramp)}
    Finally we solved together Ex.11.2 (Model Wheel Chair Ramp) from the Exercise Sheet.
    We formulated the problem as a the following GP.
    \begin{alignat*}{3}
        \min &&\qquad C_mw\ell + \frac{C_w}{2} wdh & \\ 
        \text{s.t.}&& \quad 5hd^{-1} &\leq 1 \\ 
        &&\ell w^{-1} &\leq 1 \\ 
        &&d^2\ell^{-2} + h^2\ell^{-2} &\leq 1 \\ 
        &&h &\in [0.15, 1.5] \\ 
        &&w &\in (0, 1.5] \\ 
        &&d &\in (0, 1.5] 
    \end{alignat*}
    Recall that we relaxed the last constraint with the following
    \begin{claim}{Any optimal solution of (GP) is an optimal solution for the problem (it satisfies the constraint with equality)}
        Assume we have a optimal solution $(w, h, d, \ell)$. If $d^2 + h^2= \ell^2$ then we are done. Otherwise we let 
        \begin{align*}
            \tilde{\ell}^2 := d^2 + h^2 < \ell^2
        \end{align*}
        Then $(w, h, d, \tilde{\ell})$ is still a feasible solution with cost 
        \begin{align*}
            C_mw\tilde{\ell} + \frac{C_w}{2} wdh\ <\ C_mw\ell + \frac{C_w}{2} wdh
        \end{align*}
        Therefore $(w, h, d, \ell)$ is not \underline{optimal}. \lightning
    \end{claim}
    Finally we can apply the log-log transformation to the objective and constraints and we obtain the following equivalent convex problem 
    \begin{alignat*}{3}
        \min &&\qquad \log(C_me^{W+L} + \frac{C_w}{2} e^{W+D+H}) & \\ 
        \text{s.t.}&& \quad \log(5e^{H-D}) &\leq 0 \\ 
        &&\log(e^{L - W}) &\leq 0 \\ 
        &&\log(e^{2D-2L} + e^{2H-2L}) &\leq 0 \\ 
        &&H &\in [\log(0.15), \log(1.5)] \\ 
        &&W &\in (-\infty, \log(1.5)] \\ 
        &&D &\in (-\infty, \log(1.5)] 
    \end{alignat*}
    (Recall that log-sum-exp is a convex function). After that we can return to the original variables by 
    $$ w = e^{W},\ \ d= e^{D},\ \ \ell = e^{L},\ \ h=e^{H} $$
\end{document}