\documentclass[answers]{exam}

% ------------------------------------------------------------------------------ %
% -----------------------      Base for every .tex file   ---------------------- %
% ------------------------------------------------------------------------------ %

\usepackage[dvipsnames]{xcolor}
\usepackage{mathtools}
\usepackage{amssymb}
\usepackage{amsthm}
\usepackage{amsmath}
\usepackage{framed}
\usepackage{wasysym}
\usepackage{geometry}
\usepackage{cancel}
\usepackage{blindtext}
\usepackage{pgfplots}
\usepackage{graphicx}
\usepackage{lastpage}
\usepackage[most]{tcolorbox} 
\usepackage{multicol}
\usepackage{soul}
\usepackage{listings}
\usepackage{algorithm}
\usepackage{algorithmic}
\usepackage{booktabs}
\usepackage{tikz}
\usepackage{pifont}

% Libraries
\usetikzlibrary{shapes,shapes.geometric, positioning, arrows}

\geometry{%
	left=15mm,
	right=15mm,
	top=25mm,
	bottom=25mm,
	bindingoffset=0mm,
	headheight=30pt,% output from geometry tells you what this needs to be set to as a minimum
}

% Header and Footer
\pagestyle{headandfoot}
\firstpageheadrule
\runningheadrule
\firstpageheader{Convex Optimization}{\today}{Jonathan Schnell}
\runningheader{Convex Optimization}{}{Jonathan Schnell}
\firstpagefooter{}{Page \thepage\ of \numpages}{}
\runningfooter{}{Page \thepage\ of \numpages}{}

% Commands
\newcommand{\imp}[1]{\ul{\textbf{#1}}}
\newcommand{\dproduct}[1]{\left\langle #1 \right\rangle}
\newcommand{\norm}[1]{\left\lVert #1 \right\rVert}
\renewcommand{\vector}[1]{\begin{pmatrix} #1 \end{pmatrix}}
\newcommand{\abs}[1]{\left| #1 \right|}
\newcommand{\floor}[1]{\lfloor #1 \rfloor}
\newcommand{\ceil}[1]{\lceil #1 \rceil}
\newcommand{\fracpart}[2]{\frac{\partial #1}{\partial #2}}
\newcommand{\set}[2]{\left\{#1 \ \middle|\ #2\right\}}
\renewcommand{\hat}[1]{\widehat{#1}}

\newcommand{\Ker}{\operatorname{Ker}}
\renewcommand{\Im}{\operatorname{Im}}
\renewcommand{\Re}{\operatorname{Re}}
\renewcommand{\dim}{\operatorname{dim}}
\renewcommand{\div}{\operatorname{div}}
\newcommand{\rot}{\operatorname{rot}}
\newcommand{\grad}{\operatorname{grad}}
\newcommand{\vol}{\operatorname{vol}}
\newcommand{\supp}{\operatorname{supp}}
\renewcommand{\div}{\operatorname{div}}
\newcommand*{\vertbar}{\rule[-1ex]{0.5pt}{2.5ex}}
\newcommand*{\horzbar}{\rule[.5ex]{2.5ex}{0.5pt}}

\theoremstyle{definition}
\newtheorem*{definition}{Definition}
\newtheorem*{beispiel}{Beispiel}
\newtheorem*{remark}{Remark}

\theoremstyle{plain}
\newtheorem*{proposition}{Proposition}
\newtheorem*{satz}{Satz}
\newtheorem*{korollar}{Korollar}
\newtheorem*{lemma}{Lemma}
\newtheorem*{theorem}{Theorem}


% Quote
\newtcolorbox{zitat}[1]{%
	colback=lightGray,
	grow to right by=-10mm,
	grow to left by=-10mm, 
	boxrule=0pt,
	boxsep=0pt,
	breakable,
	enhanced jigsaw,
	borderline west={4pt}{0pt}{gray},
	#1
}

% Use colors in equations
\newcommand{\highlight}[2]{\colorbox{#1}{$#2$}}%
\definecolor{lightGray}{gray}{0.9} 

% To add shortcut of script Letters in Equations
\newcommand{\s}[1]{\mathcal{#1}}
\newcommand*\circled[1]{\tikz[baseline=(char.base)]{
            \node[shape=circle,draw,inner sep=2pt] (char) {#1};}}
\newcommand{\cmark}{\ding{51}}
\newcommand{\xmark}{\ding{55}}


\newenvironment{claim}[1]{
		\par\noindent
		\textbf{Claim.} #1
		\begin{tcolorbox}[blanker, top=3mm, bottom=3mm, left=3mm, borderline west={1pt}{0mm}{black}]
		\noindent\textit{Proof of Claim.} 
}{
	\hfill$\blacksquare$	
	\end{tcolorbox}\noindent
}

% To add shortcut of number's set Z
\newcommand*{\Z}{\mathbb{Z}}
\newcommand*{\N}{\mathbb{N}}
\newcommand*{\R}{\mathbb{R}}
\newcommand*{\Q}{\mathbb{Q}}
\newcommand*{\C}{\mathbb{C}}
\newcommand*{\F}{\mathbb{F}}
\newcommand*{\K}{\mathbb{K}}

% To add shortcut of empty set
\renewcommand*{\o}{\varnothing}
\pgfplotsset{compat=1.9}

\everymath{\displaystyle}

% Line-Height
\linespread{1.15}

\graphicspath{{Files/}}

% ------------------------------

\begin{document}

	$ $
	\begin{center}
		\huge \textbf{Exercise session notes - Week 7}  \\ \vspace*{3mm}
        \Large{Theorem of Alternatives}
	\end{center}
	$ $\\

	In this week we viewed how to apply duality theory for the study of feasibility of a system involving inequalities and equalities. Consider the following system (S1)
	\begin{alignat*}{2}
		g_i(x) &\leq 0 \quad \forall i\qquad (S1) \\ 
		Ax &= b
	\end{alignat*}
	then we can check its feasibility by solving the following mathematical program (P1)
	\begin{alignat*}{2}
		\min \qquad 0& \qquad\qquad\qquad (P1)\\ 
		\text{s.t.}\quad g_i(x) &\leq 0 \quad \forall i \\ 
		Ax &= b
	\end{alignat*}
	We obtain 
	\begin{align*}
		(S1) \text{ is feasible }\iff (P1) \text{ has optimal value 0}
	\end{align*}
	Let's look at the dual program (P2) of (P1)
	\begin{alignat*}{2}
		\max \quad \hat{L}(\lambda, \nu) &= \inf_{x} \left(\sum_i \lambda_i g_i(x) + \nu^\top (Ax - b)\right) \qquad (P2) \\ 
		\text{s.t.}\quad \lambda &\geq 0
	\end{alignat*}
	If we just consider the system (S2)
	\begin{alignat*}{2}
		\hat{L}(\lambda, \nu) &> 0 \qquad (S2) \\ 
		\text{s.t.}\quad \lambda &\geq 0
	\end{alignat*}
	by weak duality we obtain 
	\begin{align*}
		(S1) \text{ is feasible }&\iff (P1) \text{ has optimal value 0} \\
		&\implies (P2) \text{ has optimal value 0} \iff (S2) \text{ is infeasible} 
	\end{align*}
	and 
	\begin{align*}
		(S2) \text{ is feasible }&\iff (P2) \text{ has optimal value >0} \\
		&\implies (P1) \text{ has optimal value }\infty \iff (S1) \text{ is infeasible} 
	\end{align*}
	Therefore the two system of inequalities (S1) and (S2) are \underline{weak alternatives} (at most one is feasible). \\ 
	If we swap inequalities with strictly inequalities we get the following systems:
	\begin{alignat*}{2}
		g_i(x) &< 0 \quad \forall i\qquad (S3) \\ 
		Ax &= b
	\end{alignat*}
	and 
	\begin{alignat*}{2}
		\hat{L}(\lambda, \nu) &\geq 0 \qquad \qquad (S4) \\ 
		\text{s.t.}\quad \lambda &\in \R^n{\geq 0},\ \lambda\neq 0
	\end{alignat*}
	Then, if there exists a point $x$ in the relative interiour of the program (P1), then the two systems are \underline{strong alternatives} (exactly one is feasible). The proof uses Slater's condition applied on an auxiliary program and convexity of $g$, and you can find it on the script (Proposition 3.7.5).


    
\end{document}