\documentclass[answers]{exam}

% ------------------------------------------------------------------------------ %
% -----------------------      Base for every .tex file   ---------------------- %
% ------------------------------------------------------------------------------ %

\usepackage[dvipsnames]{xcolor}
\usepackage{mathtools}
\usepackage{amssymb}
\usepackage{amsthm}
\usepackage{amsmath}
\usepackage{framed}
\usepackage{wasysym}
\usepackage{geometry}
\usepackage{cancel}
\usepackage{blindtext}
\usepackage{pgfplots}
\usepackage{graphicx}
\usepackage{lastpage}
\usepackage[most]{tcolorbox} 
\usepackage{multicol}
\usepackage{soul}
\usepackage{listings}
\usepackage{algorithm}
\usepackage{algorithmic}
\usepackage{booktabs}
\usepackage{tikz}
\usepackage{pifont}

% Libraries
\usetikzlibrary{shapes,shapes.geometric, positioning, arrows}

\geometry{%
	left=15mm,
	right=15mm,
	top=25mm,
	bottom=25mm,
	bindingoffset=0mm,
	headheight=30pt,% output from geometry tells you what this needs to be set to as a minimum
}

% Header and Footer
\pagestyle{headandfoot}
\firstpageheadrule
\runningheadrule
\firstpageheader{Convex Optimization}{\today}{Jonathan Schnell}
\runningheader{Convex Optimization}{}{Jonathan Schnell}
\firstpagefooter{}{Page \thepage\ of \numpages}{}
\runningfooter{}{Page \thepage\ of \numpages}{}

% Commands
\newcommand{\imp}[1]{\ul{\textbf{#1}}}
\newcommand{\dproduct}[1]{\left\langle #1 \right\rangle}
\newcommand{\norm}[1]{\left\lVert #1 \right\rVert}
\renewcommand{\vector}[1]{\begin{pmatrix} #1 \end{pmatrix}}
\newcommand{\abs}[1]{\left| #1 \right|}
\newcommand{\floor}[1]{\lfloor #1 \rfloor}
\newcommand{\ceil}[1]{\lceil #1 \rceil}
\newcommand{\fracpart}[2]{\frac{\partial #1}{\partial #2}}
\newcommand{\set}[2]{\left\{#1 \ \middle|\ #2\right\}}
\renewcommand{\hat}[1]{\widehat{#1}}

\newcommand{\Ker}{\operatorname{Ker}}
\renewcommand{\Im}{\operatorname{Im}}
\renewcommand{\Re}{\operatorname{Re}}
\renewcommand{\dim}{\operatorname{dim}}
\renewcommand{\div}{\operatorname{div}}
\newcommand{\rot}{\operatorname{rot}}
\newcommand{\grad}{\operatorname{grad}}
\newcommand{\vol}{\operatorname{vol}}
\newcommand{\supp}{\operatorname{supp}}
\renewcommand{\div}{\operatorname{div}}
\newcommand*{\vertbar}{\rule[-1ex]{0.5pt}{2.5ex}}
\newcommand*{\horzbar}{\rule[.5ex]{2.5ex}{0.5pt}}

\theoremstyle{definition}
\newtheorem*{definition}{Definition}
\newtheorem*{beispiel}{Beispiel}
\newtheorem*{remark}{Remark}

\theoremstyle{plain}
\newtheorem*{proposition}{Proposition}
\newtheorem*{satz}{Satz}
\newtheorem*{korollar}{Korollar}
\newtheorem*{lemma}{Lemma}
\newtheorem*{theorem}{Theorem}


% Quote
\newtcolorbox{zitat}[1]{%
	colback=lightGray,
	grow to right by=-10mm,
	grow to left by=-10mm, 
	boxrule=0pt,
	boxsep=0pt,
	breakable,
	enhanced jigsaw,
	borderline west={4pt}{0pt}{gray},
	#1
}

% Use colors in equations
\newcommand{\highlight}[2]{\colorbox{#1}{$#2$}}%
\definecolor{lightGray}{gray}{0.9} 

% To add shortcut of script Letters in Equations
\newcommand{\s}[1]{\mathcal{#1}}
\newcommand*\circled[1]{\tikz[baseline=(char.base)]{
            \node[shape=circle,draw,inner sep=2pt] (char) {#1};}}
\newcommand{\cmark}{\ding{51}}
\newcommand{\xmark}{\ding{55}}


\newenvironment{claim}[1]{
		\par\noindent
		\textbf{Claim.} #1
		\begin{tcolorbox}[blanker, top=3mm, bottom=3mm, left=3mm, borderline west={1pt}{0mm}{black}]
		\noindent\textit{Proof of Claim.} 
}{
	\hfill$\blacksquare$	
	\end{tcolorbox}\noindent
}

% To add shortcut of number's set Z
\newcommand*{\Z}{\mathbb{Z}}
\newcommand*{\N}{\mathbb{N}}
\newcommand*{\R}{\mathbb{R}}
\newcommand*{\Q}{\mathbb{Q}}
\newcommand*{\C}{\mathbb{C}}
\newcommand*{\F}{\mathbb{F}}
\newcommand*{\K}{\mathbb{K}}

% To add shortcut of empty set
\renewcommand*{\o}{\varnothing}
\pgfplotsset{compat=1.9}

\everymath{\displaystyle}

% Line-Height
\linespread{1.15}

\graphicspath{{Files/}}

% ------------------------------

\begin{document}

	$ $
	\begin{center}
		\huge \textbf{Exercise session notes - Week 12}  \\ \vspace*{3mm}
        \Large{Formulations as SDP + Schur Complement}
	\end{center}
	$ $\\

    Consider a quadratic program of the form
    \begin{align*}
        \min\quad c^\top x &\\
        x^\top Qx + b^\top x + d &\leq 0 
    \end{align*}
    If $Q$ is positive semi-definite, then this problem is convex and we can solve it efficiently via Newton Descent. But what if $Q$ has negative eigenvalues? Then the problem is no longer convex, but we can still solve it partially with a relaxation to a SDP. We first define the matrix $X = xx^\top$ which is a symmetric matrix and contains $x_ix_j$ on the $i$th row and $j$th column. This is often used when we have a quadratic form because then 
    $$ x^\top Q x = \sum_{i,j} x_i Q_{ij} x_j = \sum_{i,j} X_{ij} Q_{ij} = \operatorname*{Tr}(XQ) = \dproduct{X, Q} $$
    and this is an affine function on $S_+^n$. Therefore our QP is equivalent to 
    \begin{align*}
        \min \quad c^\top x& \\ 
        \dproduct{X,Q} + b^\top x + d &\leq 0 \\ 
        X &= xx^\top
    \end{align*}
    Finally we can relax the last constraint and we get the SDP
    \begin{align*}
        \min \quad c^\top x& \\ 
        \dproduct{X,Q} + b^\top x + d &\leq 0 \\ 
        X&\succeq 0
    \end{align*}
    Note that by solving this problem we obtain a lower bound on the optimal value of the QP. \\ 

    A very important fact on positive semi-definite matrices is the Schur Complement: 
    \begin{proposition}[Schur Complement]
        Let $X$ be a symmetric block matrix 
        $$ X = \begin{pmatrix}
            A & B \\ B^\top & C
        \end{pmatrix} $$
        If $A\succ 0$ then 
        $$ X\succeq 0 \iff C-B^\top A^{-1}B \succeq 0 $$
        If $C\succ 0$ then 
        $$ X\succeq 0 \iff A-B C^{-1}B^\top \succeq 0 $$
    \end{proposition}
    Consider the following $0$-$1$ QP
    \begin{align*}
        \max\quad y^\top B y& \\
        y &\in  \{0,1\}^n
    \end{align*}
    We define as before $Y = y y^\top$ and note that 
    $$ y_i^2 = y_i \implies \operatorname*{diag}(Y) = y $$
    In particular we can write 
    $$ Y = \operatorname*{diag}(Y) \operatorname*{diag}(Y)^\top $$
    We can relax this constraint into the following 
    $$ Y - \operatorname*{diag}(Y) \operatorname*{diag}(Y)^\top \succeq 0 $$
    and by the Schur Complement we can write a relaxation of the QP program as the SDP 
    \begin{align*}
        \max \quad\dproduct{Y, B}& \\ 
        \begin{pmatrix}
            Y & \operatorname*{diag}(Y) \\ \operatorname*{diag}(Y)^\top & 1
        \end{pmatrix} &\succeq 0 \\ 
        Y&\succeq 0
    \end{align*}
    As before, by solving this problem we obtain a lower bound on the optimal value of the $0$-$1$ QP. \\ 

    Another application of the Schur Complement comes from geometry: We are given $k$ points $z_1, \ldots, z_k\in \R^n$ and we want to find an Ellipse $E$ that contains all the points and has minimal volume. We can describe an Ellipse as 
    $$ E_{P, c} = \set{x}{(x-c)^\top P(x-c)} $$
    where $c\in \R^n$ is the center of the Ellipse and $P\succ 0$ is the matrix that gives us the stretch of the Ellipse (if $P=I$ then we obtain a circle). Moreover the volume of the Ellipse is proportional to 
    $$ \vol(E_{P,c}) \sim \sqrt{\operatorname*{det}(P^{-1})} $$
    Therefore we can model this problem as the following abstract program
    \begin{alignat*}{3}
        \min_{P,c} &&\quad \operatorname*{vol}(E_{P,c}) &\sim \sqrt{\operatorname*{det}(P^{-1})} \\ 
        \text{s.t.}&&\quad  z_i &\in E_{P,c} \qquad \forall i
    \end{alignat*}
    Since the function $\log(\cdot)$ is monotone, we can apply that to the objective function and the problem remains the same. 
    \begin{alignat*}{3}
        \min_{P,c} &&\quad \log\left(\sqrt{\operatorname*{det}(P^{-1})}\right) &= -\frac{1}{2} \log\det(P) \\ 
        \text{s.t.}&&\quad  (z_i-c)^\top P(z_i - c) &\leq 1 \qquad \forall i
    \end{alignat*}
    Recall that $\log\det(\cdot)$ is a concave function. Since $P\succ 0$ we can write it as $P = B^\top B$ and we get 
    \begin{alignat*}{3}
        \min_{B, c} &&\quad -\frac{1}{2} \log\det(B^\top B) &= -\log\det(B) \\ 
        \text{s.t.}&&\quad  1- (z_i-c)^\top B^\top B(z_i - c) &\geq 0 \qquad \forall i
    \end{alignat*}
    Finally, we can define $d := Bc$ and apply the Schur Complement to the constraint so that obtain the SDP 
    \begin{alignat*}{3}
        \min_{B, d} &&\quad -\log\det(B)& \\ 
        \text{s.t.}&&\quad \begin{pmatrix}
            I & Bz_i - d \\ 
            (Bz_i - d)^\top & 1
        \end{pmatrix}&\succeq 0 \qquad \forall i
    \end{alignat*}
    After we have solved this program we can obtain a solution to the original problem by computing $P = B^\top B$ and $c = B^{-1}d$.
    

\end{document}